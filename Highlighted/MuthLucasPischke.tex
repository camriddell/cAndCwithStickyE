  \section{Muth--Lucas--Pischke and \cite{reis:inattentive}} \label{sec:Comparisons}

  Now that our calibrations and results have been presented, we are in position to make some quantitative comparisons of our model to two principal alternatives to habit formation (or our model) for explaining excess smoothness in consumption growth, by Pischke and by Reis.

  \subsection{Muth--Lucas--Pischke}
  The longest-standing rival to habit formation as an explanation of consumption sluggishness is what we will call the Muth--Lucas--Pischke (henceforth, MLP) framework.  The idea is not that agents are inattentive, but instead that they have imperfect information on which they perform an optimal signal extraction problem.

  \cite{muthOptimal}'s agents could observe only the level of their income, but not the split between its permanent and transitory components.  He derived the optimal (mean-squared-error-minimizing) method for estimating the level of permanent income from the observed signal about the level of actual income.  \cite{lucas:imperfectInfo} applied the same mathematical toolkit to solve a model in which firms are assumed to be unable to distinguish idiosyncratic from aggregate shocks.  \cite{pischkeMicroMacro} combines the ideas of Muth and Lucas and applies the result to micro consumption data: His consumers have no ability at all to perceive whether income shocks that hit them are aggregate or idiosyncratic, transitory or permanent.  They see only their income, and perform signal extraction on it.

  Pischke calibrates his model with micro data in which he calculates that transitory shocks vastly outweigh permanent shocks.\footnote{Pischke's estimates constructed from the {\it Survey of Income and Program Participation} are rather different from the magnitudes of transitory and permanent shocks estimated in the extensive literature---mostly subsequent to Pischke's paper---cited in our calibration section above.}  So, when a shock arrives, consumers always interpret it as being almost entirely transitory and change their consumption by little.  However, macroeconometricians have long known that {\it aggregate} income shocks are close to permanent.  When an aggregate permanent shock comes along, Pischkian consumers spend very little of it, confounding the aggregate permanent shock's effect on their income with the mainly transitory idiosyncratic shocks that account for most of the total variation in their income.  This misperception causes sluggishness in {\it aggregate} consumption dynamics in response to aggregate shocks.

  In its assumption that consumers fail to perceive aggregate shocks immediately and fully, Pischke's model resembles ours.  However, few papers in the subsequent literature have followed Pischke in making the assumption that households have no idea, when an idiosyncratic income shock occurs, whether it is transitory or permanent.  Especially in the last decade or so, the literature instead has almost always assumed that consumers can perfectly perceive the transitory and permanent components of their income; \hyperlink{Why-Consumers-See-Individual-Shocks}{see our defense of this assumption above}.

  Granting our choice to assume that consumers correctly perceive the events that are idiosyncratic to them (job changes, lottery winnings, etc), there is still a potential role for application of the MLP framework:  Instead of assuming sticky expectations, we could instead have assumed that consumers perform a signal extraction exercise on \textit{only} the aggregate component of their income, because they cannot perceive the transitory/permanent split for the (tiny) part of their income change that reflects aggregate macroeconomic developments.

  In principle, such confusion could generate excess smoothness; for a detailed description of the mechanism, \hyperlink{MuthLucasPischke}{see} online Appendix~\ref{appendix:Muth}.  But, defining the signal-to-noise ratio $\varphi=\sigma^2_{\Psi}/\sigma^2_{\Theta}$, Muth's derivations imply that the optimal updating coefficient is:
  \input{\eq/muthOptimal.tex}
  \providecommand{\PischkePi}{0.83}
  \providecommand{\PischkePiCancel}{0.17}
  \providecommand{\fromFile}{false}
  \providecommand{\FileOrNot}{\ifthenelse{\boolean{\fromFile}}}
  Plugging our calibrations of $\sigma^2_{\Psi}$ and $\sigma^2_{\Theta}$ from section \ref{sec:calibration} into \eqref{eq:muthOptimal}, the model yields a predicted value of \FileOrNot{$1-\Pi \approx \PischkePiCancel$}{$(1-\Pi) \approx \PischkePiCancel $}---very far below the approximately $0.6$ estimate from \cite{hrsHabit} and even farther below our estimate of roughly $0.7$--$0.8$ for U.S.\ data.  This reflects the well-known fact that aggregate income is hard to distinguish from a random walk; if it were perceived to be a perfect random walk with no transitory component at all, the serial correlation in its growth would be zero.  So, in practice, allowing signal extraction with respect to the aggregate data is not a path to explaining excess smoothness.

  \subsection{\cite{reis:inattentive}}

  Leaving aside our earlier criticisms of its fidelity to microeconomic evidence, the model of \cite{reis:inattentive} has a further disadvantage relative to any of the other three stories (habits, MLP, or our model) with respect to aggregate dynamics. In Reis's model consumers update their information on a regular schedule---under a plausible calibration of the model, once a year. One implication of the model is that the change in consumption at the next reset is unpredictable; this implies that aggregate consumption growth would be unpredictable at any horizon beyond, say, a year.\footnote{In contrast, our model exhibits significant predictability beyond one year. The value of $\chi$ in the `horse-race' regression for the SOE economy is 0.66 when the right hand side is lagged by one quarter (see Table~\ref{tPESOEsim}). Adding an extra one and two years' lag to the right hand side sees $\chi$ decline approximately as an AR(1), to 0.20 and 0.06 respectively.}  But, macroeconomists felt compelled to incorporate sluggishness into macroeconomic models in large part to explain the fact that consumption growth is forecastable over extended periods---empirical impulse response functions indicate that a macroeconomically substantial component of the adjustment to shocks takes place well beyond the one year horizon.  A calibration of the Reis model in which consumers update once a year therefore fails to solve a large part of the original problem (of medium-term predictability).

